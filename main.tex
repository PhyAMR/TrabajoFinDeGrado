\documentclass[11pt, a4paper]{article} %tamaño mínimo de letra 11pto.

\usepackage{graphicx} 
\usepackage[spanish]{babel} %Español 
\usepackage[utf8]{inputenc} %Para poder poner tildes
\usepackage{vmargin} %Para modificar los márgenes
\setmargins{2.5cm}{1.5cm}{16.5cm}{23.42cm}{10pt}{1cm}{0pt}{2cm}
%margen izquierdo, superior, anchura del texto, altura del texto, altura de los encabezados, espacio entre el texto y los encabezados, altura del pie de página, espacio entre el texto y el pie de página
\usepackage{hyperref}
\begin{document}
%%%%%%Portada%%%%%%%
\begin{titlepage}
\centering
{ \bfseries \Large UNIVERSIDAD COMPLUTENSE DE MADRID}
\vspace{0.5cm}

{\bfseries  \Large FACULTAD DE CIENCIAS FÍSICAS} 
\vspace{1cm}

{\large DEPARTAMENTO DE FÍSICA DE LA TIERRA Y ASTROFÍSICA}
\vspace{0.8cm}

%%%%Logo Complutense%%%%%
{\includegraphics[width=0.35\textwidth]{logo_UCM.png}} %Para ajustar la portada a una sola página se puede reducir el tamaño del logo
\vspace{0.8cm}

{\bfseries \Large TRABAJO DE FIN DE GRADO}
\vspace{2cm}

{\Large Código de TFG:  [C\'odigo TFG] } \vspace{5mm}

{\Large [Relaciones estructurales de galaxias remotas a partir de los catálogos CANDELS]}\vspace{5mm}

{\Large [Structural relations of remote galaxies from the CANDELS catalogues]}\vspace{5mm}

{\Large Supervisor/es: [Nombre del/os supervisores]}\vspace{20mm} 

{\bfseries \LARGE Jesús Gallego Maestro}\vspace{5mm} 

{\large Grado en Física}\vspace{5mm} 

{\large Curso acad\'emico 20[24-25]}\vspace{5mm} 

{\large Convocatoria XXXX}\vspace{5mm} 

\end{titlepage}
\newpage

%{\bfseries \large [Título extendido del TFG (si procede)] }\vspace{10mm} 

{\bfseries \large Resumen:} \vspace{5mm}

La busqueda de patrones en los datos de experimentos cientificos es una de las principales
fuentes de informacion en el campo de la astrofísica. El universo a grandes escalas está 
compuesto por procesos muy complejos, por lo que la
observación de estos patrones nos permite entender mejor el funcionamiento de estos procesos.

Durante muchos años, se ha observado el firmamento con nuevas y mejores tecnologías con el 
fin de tener una mayor cantidad de datos de los que obtener información. El catálogo CANDELS
contiene una gran cantidad de datos sobre galaxias,  a varios redshifts diferentes, que 
contiene una gran cantidad de datos sobre propiedades físicas acercas de estas. 

El objetivo es usar este catálogo para analizar los datos con un algoritmo de regresión simbólica 
para obtener relaciones ya conocidas, y ver que otras relaciones se pueden obtener a partir de 
estos datos.

[Rsumen de lo encontrado]
[Conclusiones]
[Prespectiva]
\vspace{10mm}

{\bfseries \large Abstract: } \vspace{5mm} 

The search of patterns in scientific experiments data is one of the main sources of 
information in astrophysics. The vast universe is characterized of complex, that makes the
observation of these patterns can help us to understand better the working of these processes.

For many years, the cosmic dust has been observed with new and better technologies to obtain a 
higher amount of data. The CANDELS catalogue contains a large amount of data about galaxies, 
at different redshifts, that contains a large amount of data about physical properties of these objects.

The goal is to use this catalogue to analyze the data with a symbolic regression algorithm to obtain 
known relations, and see that other relations can be obtained from these data.

[Abstract of what has been found]
[Conclusions]
[Outlook]

\vspace{1cm}

%%Comentar estas notas para que no salgan en la memoria
%{\Large\textbf{Nota: el título extendido (si procede), el resumen y el abstract deben estar en una misma página y su extensión no debe superar una página. Tamaño mínimo 11pto.}}
%\vspace{1cm}

%{\Large\textbf{Extensión máxima 20 páginas sin contar portada ni resumen (sí se incluye índice, introducción, conclusiones y bibliografía}}
\newpage

%%Inicio:
\tableofcontents


\section{Introducción}
\section{La exploración CANDELS}
\section{Análisis de relaciones estructurales}
\subsection{Análisis de relaciones estructurales 2D}
\subsection{Análisis de relaciones estructurales 3D}
\subsection{Regresión simbólica}
La regresión simbólica (SR) es un tipo de ajuste a los datos en el que no se parte de un 
modelo concreto, como puede ser la regresión lineal o ajustar a una exponencial, sino que 
se parte de una serie de operadores o funciones, como puede ser el uso de funciones trigonométricas 
o usar funciones con exponentes, que busca el mejor modelo como combinación de esos operadores.  

En este caso concreto se ha usado la librería \href{https://github.com/MilesCranmer/PySR}{PySR} la cual
ha sido elegida por su impletenciación en Python, por los ''benchmarks''  o resultados que se describen en \cite{cranmerInterpretableMachineLearning2023}
\subsubsection{Fundamento}
\section{Resultados principales}
\section{Conclusiones}

\bibliographystyle{plain} % We choose the "plain" reference style
\bibliography{refs} % Entries are in the refs.bib file
\end{document}
